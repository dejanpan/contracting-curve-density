%%% Local Variables: 
%%% mode: latex
%%% TeX-master: "../main"
%%% End: 
\chapter{Related work}
\label{chapter:related}

As mentioned in the previous chapter, the
CCD approach introduced in this chapter is a model-based computer
vision algorithm. In the field of
computer vision, the topics of model-based approach have been addressed in numerous research papers.
In this chapter we review selected publications related to the topics
covered in this thesis and try to reveal the evolution of algorithms. The
articles have been carefully selected based on the following criteria:
\begin{itemize}
\item Articles on 2-Dimensional and 3-Dimensional deformable models,
  such as snakes and gradient vector flow deformable models~\cite{xu2000gradient}.
\item Articles on applying statistical knowledge to the models
\end{itemize}
There are possible overlaps in the two categories. We will discuss the
advantages and limitations of these methods related to the work we aim
to solve. In section~\ref{2d3ddm} we will discuss the related work
belonging to the first category. This is followed by the introduction
of previous work on statistical models.
% The topics of active contour, the Contracting
% Curve Density approach and model tracking have been addressed in
% numerous research papers.

\section{2-Dimensional and 3-Dimensional deformable model}
\label{sec:2d3ddm}
Many traditional segmentation methods are effected by the assumption that the
images studied in computer vision are usually self-contained, namely,
the information needed for a successful segmentation can be extracted
from the images.

In 1980s, a paradigm named \textit{Active Vison}~\cite{aloimonos1988active} escaped this bind and
pushed the vision process in a more goal-directed fashion. After that, a
notably successful departure, the \textit{Snake}, is proposed in a
seminar work conducted by Kass~\cite{kass1988snakes}. The original paper~\cite{kass1988snakes}, spawned many variations
and extensions including the use of Fourier parameterisation~\cite{scott1987alternative}, incorporation
of a topologically adaptable model~\cite{mcinerney1995topologically} and
its application~\cite{mcinemey1999topology}, incorporation of a discrete
dynamic contour model~\cite{lobregt1995discrete}. A realization of snakes
using B-splines was developed in~\cite{brigger2000b}. In two
dimensions, The snake has a variation named active shape
model~\cite{cootes1995active}, which is a discrete version of this
approach. Gradient Vector Flow~\cite{xu1998snakes}, or GVF, is an extension developed
based on a new type of external field. Another paper written by the same
authors~\cite{xu2000gradient} is concerned with the convergence properties of
deformable models. In three dimensions, a good deal of research work has
been conducted on matching three-dimensional models, both on rigid
~\cite{harris1993tracking} and deformable~\cite{terzopoulos1991dynamic} shapes. 

Besides the segmentation,
dynamic models, such as snake and its variations, are greatly used in application of object
tracking. A real-time tracking system based on deformable models, such
as snakes, is developed
in~\cite{terzopoulos1992tracking}. It proved that the active shape and
motion estimators are able to deal very effectively with the complex
motions of nonrigid objects. Furthermore, the combination of active
models and Kalman filter theory is also a popular approach to
tracking, some work about this can be found
in~\cite{schick1991simultaneous}. And last but not least,
~\cite{blake1998active} is a complete volume about the topics of
geometric and probabilistic models  for shapes and their dynamics.

\section{Statistical models}
\label{sec:sm}
Pattern recognition theory is a general statistical framework which is
important in the study of model-based approaches. This has started from the 1970s
and 1980s when a new interpretation of image was proposed in the
statistical community.  Analyzing the model problems in probabilistic
context has two great advantages. The first is that it approaches the
nature of the problems, the ranges of shapes are defined by a
probability, this provides another viewpoint for curve fitting problem
in the field of computer vision. Another advantage is that after we
convert a problem in pattern recognition, there are abundant tools and
skills to deal with such a problem. 
The CCD approach is a method developed in the probabilistic context. 

In~\cite{kelemen1999three} and~\cite{kelemen1999elastic}, an elegant
use of statistical models for the segmentation of medical images is
designed.  The resulting segmentation system consists of building
statistical models and automatic segmentation of new image data
sets by restricting elastic deformation of models.  The work
in~\cite{sclaroff2001deformable} and~\cite{liu1999deformable} also
exploit the prior knowledge from the perspective of probability,
furthermore, the statistical shape models enforce the prior
probabilities on objects by designing a complicated energy function.  In this thesis, we assume that the shapes' priors have a Gaussian form in
shape-space. In the case of a norm-squared density over  quadratic spline space,
the prior is a Gaussian Markov Random Field (MRF)~\cite{blake1998active}, which is used widely  for modelling prior
distributions for curves~\cite{storvik1994bayesian}.


However, defining a prior distribution for shape is only part of the
problem. Prior knowledge only controls the feature interpretation in an
image. A successful segmentation or curve fitting requires
obtaining the posterior distribution, which takes account of what
shape is actually likely to be present in a particular image. Inspired
by the iterative algorithm k-means~\cite{ding2004k} and
Expectation-Maximization (EM)~\cite{dempster1977maximum} algorithm, especially the later one, a novel
curve fitting method the CCD algorithm is proposed in
~\cite{hanek2004contracting}. Bayesian treatment is used in this
algorithm for incorporation of maximum a posteriori probability (MAP)
~\cite{sorenson1980parameter} estimate instead of a complicated cost
function. Unlike the EM algorithm, the CCD uses local statistics
instead of global statistics. Moreover, blurred curve models is
proposed as efficient means for iteratively optimizing. The algorithm
can be used in object localization and object tracking. As an example
of applications of the CCD approach, an efficient, robust and fully
automatic real-time system for 3D object pose tracking in
image sequences is presented in~\cite{panin2006fully} and~\cite{panin2006efficient}. MultiOcular Contracting Curve Density
algorithm (MOCCD)~\cite{hahn2007tracking} is an extension of the CCD
approach. In the paper, it is integrated into the tracking system of the human body. 

Both the snake and CCD can be used to build naive tracking
system. However, they are limited by the performance and stability
problems. Several methods achieve a speed-up by propagating a Gaussian
distribution of the model parameters over time, such as tracking based
Kalman filter in~\cite{brookner1998tracking}. The method is limited by the range of probability
distributions they presented. The condensation algorithm (Conditional
Density Propagation)~\cite{isard1998icondensation} is proved as a marked improvement in tracking performance. Another feature of
this method is that it only considers the pixels on some
perpendiculars. The CCD tracker~\cite{hanek2004fitting} uses only
pixels on some perpendiculars like the condensation algorithm, but
focuses on the vicinity of the contour. This means the CCD tracker can
save time and improve performance.

In the CCD algorithm and its variations, the curve fitting probelm is
often converted to an optimization one. The optimization step is
outstanding important for the CCD approach. There are many methods to
deal with optimization, which can be classified into two
categories, one is that global optimization and another is the local
optimization~\cite{hanek2004fitting}. The later one is used in the CCD
approach. First, a smoothed objective function is obtained by fitting
the curve model to a large scale description. Then the window's size is
gradually reduced. During the process, many types of  numerical
optimization  methods such as  conjugate gradient method , Newton's
method, Gaussian-Newton and Levenberg-Marquardt algorithm~\cite{contourpanin2011} can be used.


