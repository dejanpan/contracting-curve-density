%%% Local Variables: 
%%% mode: latex
%%% TeX-master: "../main"
%%% End: 
\chapter{Related Work}
\label{chapter:related}

As mentioned in the previous chapter, the
CCD approach introduced in this thesis is a model-based computer
vision algorithm. In the field of
computer vision, the topics of model-based approach have been addressed in numerous research papers.
In this chapter we review selected publications related to the topics
covered in this thesis and try to reveal the evolution of algorithms. The
articles have been carefully selected based on the following criteria:
\begin{itemize}
\item Articles on two-dimensional and three-dimensional deformable models,
  such as Snakes and gradient vector flow deformable models.
\item Articles on applying statistical knowledge to the models
\end{itemize}
Currently, the second category becomes very popular. The CCD algorithm proposed in
~\cite{hanek2004contracting} is inspired from those methods for solving
supervised learning problems (e.g. regression, classification) in
the context of probability.

In section~\ref{sec:2d3ddm} we will discuss the related work
belonging to the first category. In the last section, we briefly
introduce the OpenCV library.ie 
% The topics of active contour, the Contracting
% Curve Density approach and model tracking have been addressed in
% numerous research papers.

\section{Two-dimensional and Three-dimensional Deformable Models}
\label{sec:2d3ddm}

\subsection{Two-dimensional \& Three-dimensional Models}
\label{sec:23m}
Many traditional segmentation methods are effected by the assumption that the
images studied in computer vision are usually self-contained, namely,
the information needed for a successful segmentation can be extracted
from the images.

In 1980s, a paradigm named \textit{Active Vision}~\cite{aloimonos1988active} escaped this bind and
pushed the vision process in a more goal-directed fashion. After that, a
notably successful departure, the \textit{Snakes}, is proposed in a
seminar work conducted by Kass~\cite{kass1988snakes}. The original paper, spawned many variations
and extensions including the use of Fourier
parameterisation~\cite{scott1987alternative}, and incorporation
of a topologically adaptable models~\cite{mcinerney1995topologically},
thereof application~\cite{mcinemey1999topology} and incorporation of a discrete
dynamic contour model~\cite{lobregt1995discrete}. A realization of the
Snakes using B-splines was developed in~\cite{brigger2000b}. In two
dimensions, the Snakes have a variation named active shape
model~\cite{cootes1995active}, which is a discrete version of this
approach. Gradient Vector Flow~\cite{xu1998snakes}, or GVF, is an extension developed
based on a new type of external field. In~\cite{xu2000gradient}
authors are concerned with the convergence properties of deformable
models. In three dimensions, a good deal of research work has been
conducted on matching three-dimensional models, both on rigid
~\cite{harris1993tracking} and deformable~\cite{terzopoulos1991dynamic} shapes.
\subsection{Appliocations}
\label{sec:app}
Model-based segmentation methods are widely used in the field of
medical image processing. Besides the segmentation,
dynamic models, such as the Snakes and its variations, are greatly used in application of object
tracking. A real-time tracking system based on deformable models, such
as the Snakes, is developed
in~\cite{terzopoulos1992tracking}. It proved that the active shape and
motion estimators are able to deal very effectively with the complex
motions of nonrigid objects. Furthermore, the combination of active
models and Kalman filter theory is also a popular approach to
tracking, some work about this can be found
in~\cite{schick1991simultaneous}. And last but not least,
~\cite{blake1998active} is a complete volume about the topics of
geometric and probabilistic models  for shapes and their dynamics.

\section{Statistical Models}
\label{sec:sm}
Pattern recognition theory is a general statistical framework which is
important in the study of model-based approaches. This has started from the 1970s
and 1980s when a new interpretation of image was proposed in the
statistical community.  Analyzing the model problems in probabilistic
context has two great advantages. The first is that it approaches the
nature of the problems, the ranges of shapes are defined by a
probability, this provides another viewpoint for the curve-fitting problem
in the field of computer vision. Another advantage is that when we
solve the problem in the field of pattern recognition, there are abundant tools to deal with such problems. 
The CCD approach is a method developed in the probabilistic context. 

In~\cite{kelemen1999three} and~\cite{kelemen1999elastic}, an elegant
use of statistical models for the segmentation of medical images is
designed.  The resulting segmentation system consists of building
statistical models and automatic segmentation of new image data
sets by restricting elastic deformation of models.  The works
in~\cite{sclaroff2001deformable} and~\cite{liu1999deformable} also
exploit the prior knowledge from the perspective of probability,
furthermore, the statistical shape models enforce the prior
probabilities on objects by designing a complicated energy function.  In this thesis, we assume that the shapes' priors have a Gaussian form in
shape-space. In the case of a norm-squared density over  quadratic spline space,
the prior is a Gaussian Markov Random Field
(\acronym{MRF}{Markov Random Field})~\cite{blake1998active}, which is used widely  for modeling
prior distributions for curves~\cite{storvik1994bayesian}.

Defining a prior distribution for shape is only part of the
problem, prior knowledge only controls the feature interpretation in an
image, also it just approximates the contour of an observed object. In
order to solve the problem, likelihood function is required. In some
special cases, we can get a solution by maximizing likelihood, but
usually it is intractable because there is no closed-solution. An indirect
approach known as iterative reweighted least squares
(IRLS)~\cite{bishop2006pattern} is used find the parameters of the model.
the CCD  algorithm uses local statistics to evaluate a conditional distribution
, then the iterative maximum a posteriori probability (MAP)
~\cite{sorenson1980parameter} estimate process is used to refine
parameters instead of maximizing a complicated cost function. 
Moreover, a blurred curve model is proposed as a efficient mean for iteratively optimizing. The algorithm
can be used in object localization and object tracking. As an example
of applications of the CCD approach, an efficient, robust and fully
automatic real-time system for 3D object pose tracking in
image sequences is presented in~\cite{panin2006fully}
and~\cite{panin2006efficient}. MultiOcular Contracting Curve Density
algorithm (\acronym{MOCCD}{MultiOcular Contracting Curve Density})~\cite{hahn2007tracking} is an extension of the CCD
approach. In the paper, it is integrated into the tracking system of
the human body. 
% A successful segmentation or curve fitting requires to
% construct the posterior distribution, which accounts for what
% shape is actually likely to be present in a particular image. 
 % Bayesian treatment is used for incorporation of 
% Unlike the EM algorithm, the CCD uses local statistics
% instead of global statistics.

Both the Snakes and CCD can be used to build naive tracking
system. However, they are limited by the performance and stability
problems. Several methods achieve a speed-up by propagating a Gaussian
distribution of the model parameters over time, such as tracking based
Kalman filter in~\cite{brookner1998tracking}. The method is limited by the range of probability
distributions they presented. The \acronym{condensation}{Conditional Density Propagation} (Conditional
Density Propagation)~\cite{isard1998icondensation} is proved as a marked improvement in tracking performance. Another feature of
this method is that it only considers the pixels on some
perpendiculars of a contour. The CCD tracker~\cite{hanek2004fitting} uses only
pixels on some perpendiculars like the condensation algorithm, but
focuses on the vicinity of the contour. This means the CCD tracker can
save time and improve performance.

In the CCD algorithm and its variations, the curve-fitting process is
often addressed in an optimization stage. The optimization step is
very important for the CCD approach. There are many methods to
deal with optimization, which can be classified into two
categories, one is that global optimization and another is the local
optimization. The latter one is used in the CCD
approach and it works as follows: First, a smoothed objective function is obtained by fitting
the curve model to a large scale description. Then the window's size is
gradually reduced. During the process, many types of  numerical
optimization methods such as  conjugate gradient method , Newton's
method, Gaussian-Newton and Levenberg-Marquardt
(\acronym{LMM}{Levenberg-Marquardt method})
algorithm~\cite{contourpanin2011}, least squares support vector
machine (\acronym{LS-SVM}{least squares support vector
machine})~\cite{vapnik2000nature} can be used.
