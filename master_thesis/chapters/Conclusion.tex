%%% Local Variables: 
%%% mode: latex
%%% TeX-master: "../main"
%%% End: 

\chapter{Conclusion and  Future Work}
\label{chapter:conclusion}

\section{Conclusion}
\label{sec:con}

In this work, we have investigated the Contracting Curve Density (CCD)
algorithm and its application on personal robotics. Starting from
parametric model curve, we have analyzed the core of the algorithm and
proposed some improvement and optimization on it. Then we apply the
algorithm on some challenging computer vision problems, such as curve
fitting, object segmentation and tracking, 3-D pose estimation.  We
delivered some experiments on PR2 and show satisfactory results. 

The CCD algorithm is a iterative procedure to refine an
a prior distribution of the model parameters to a Gaussian
approximation of the posterior distribution. It aims to extract the
contour of a object based a prior model. A MAP estimate need to be
evaluated by applying an effective local optimization method in order to get
the mean of model parameters and covariance matrices which govern the
posterior distribution and determine the shape of model. Likelihood is 
obtained from local statistics of the vicinity of the expected
curve. The local statistics provides locally adapted criteria for
adjacent image regions seperated by a contour. This criteria is stable
and effective than those relying on homogeneous image regions or
specific edge properties. The local statistics is outstanding
important for the CCD algorithm, in this thesis we design a new weight
function as criteria for separating adjacent image regions. 
 In order to improve the performance and
reduce computing cost, a blurred curve model
\cite{hanek2004contracting} is applied as an efficient means for
iteratively optimization. In the iteration process, after the
algorithm starts to converge, the area of
convergence becomes smaller and smaller, this has two great
advantages, one is that the pixels used to learn the local statistic
are scaled, thus computing cost is decreased; Another one is that it
leads a high level sub-pixel accuracy. The rate of decreasing of the
convergence area is significant, if too fast, the algorithm will stop
converge, on the other hand, if too slow, it is difficult to achieve
notably performance improvement. We investigate the relation between
the rate and the convergence speed in this thesis.
Beside, some features and improvement are integrated in to the
developed system.
\begin{itemize}
\item model the parametric curve is implemented using the quadratic or
cubic B-spline curve  to avoid the Runge phenomenon without increasing
the degree.
\item support both planar affine and non-planar affine
  shape-space. This leads that the system is viewpoint invariant.
\item Full automatic initialization of contour is developed to avoid
  intervention from human.
\item Initialization from point cloud.
\end{itemize}

Due to the robustness, stability and high performance, one main
application of the CCD algorithm is on the development of real-time
object tracking system. The CCD tracker is presented and developed in
this thesis, and simulation and experimental of results have been presented, through a
stand-alone implementation running both on
standard hardware and PR2.

We demonstrate the experimental results and compare it with other
segmentation or tracking algorithm from some aspects such as
robustness, performance and accuracy. quantitative and qualitative
results are presented in this thesis.

The experimental analysis shows that the CCD approach is capable of
achieving high sub-pixel accuracy and robustness even in the presence
of heavy texture, clutter, partial occlusion and severe changes of
illumination. It could be used in  visual-guided robot
manipulation, robot navigation (self-localization) and target
tracking.

\section{Feature work}
\label{sec:feature}
We can improve the CCD algorithm from many aspects in order to make it
more stable on personal robotics.

Future developments of the system include the integration
into a more complex tracking framework, that
will allow more complex applications like as body parts
(faces, hands) or full-body tracking, automatic assembly
procedures for mechanical parts, and many others as well
\begin{itemize}
\item
\item
\item
\item
\item
\item
\item
\item
\item
\item
\item
\item 
\end{itemize}

Moreover, Some possible application of the CCD approach could be
investigated.
\begin{itemize}
\item
\item
\item
\item
\item
\item
\item
\item
\item
\item
\item
\item
\item 
\end{itemize}
