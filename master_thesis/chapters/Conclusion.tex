%%% Local Variables: 
%%% mode: latex
%%% TeX-master: "../main"
%%% End: 

\chapter{Conclusion and  Future Work}
\label{chapter:conclusion}

\section{Conclusion}
\label{sec:con}

In this work, we have investigated the Contracting Curve Density (CCD)
algorithm and its application on personal robotics. Starting from
parametric model curve, we have analyzed the core of the algorithm and
proposed some improvements and optimizations on it. Then we applied the
algorithm on some challenging computer vision problems, such as curve
fitting, object segmentation and tracking, 3-D pose estimation. We
delivered some experiments on the PR2 robot which are showing satisfactory results. 

The CCD algorithm is an iterative procedure to refine an
a prior distribution of the model parameters to a Gaussian
approximation of the posterior distribution. It aims to extract the
contour of an object based a prior model. A MAP estimate need to be
evaluated by applying an effective local optimization method in order to get
the mean of model parameters and covariance matrices which govern the
posterior distribution and determine the shape of model. Likelihood is 
obtained from local statistics of the vicinity of the expected
curve. The local statistics provides locally adapted criteria for
adjacent image regions seperated by a contour. This criteria is stable
and effective than those relying on homogeneous image regions or
specific edge properties. The local statistics are very
important for the CCD algorithm, in this thesis we design a new weight
function as criteria for separating adjacent image regions. 
 In order to improve the performance and
reduce computing cost, a blurred curve model~\cite{hanek2004contracting} is applied as an efficient means for
iteratively optimization. In the iteration process, after the
algorithm starts to converge, the area of
convergence becomes smaller and smaller. This has two great
advantages, one is that the pixels used to learn the local statistic
are scaled, thus computing cost is decreased. Another one is that it
leads a high level sub-pixel accuracy. The decreasing rate of the
convergence area is significant, if too fast, the algorithm will stop
converging, on the other hand, if too slow, it is difficult to achieve
notably performance improvement. In this thesis we have investigated the relation between
the rate and the convergence speed.
Beside, some features and improvements are integrated in to the
developed system.
\begin{itemize}
\item Using the logistic sigmoid function instead of Gaussian error
  function to convert a curve-fitting problem to a Gaussian logistic
  regression problem in the field of pattern recognition.
\item Modeling the parametric curve is implemented using the quadratic or
cubic B-spline curve  to avoid the Runge phenomenon without increasing
the degree.
\item Supports both planar affine and non-planar affine
  shape-space. This leads that the system is viewpoint invariant.
\item Full automatic initialization of contour is developed to avoid
  intervention from human.
\item Initialization from point cloud.
\end{itemize}

Due to the robustness, stability and high performance, one main
application of the CCD algorithm is the development of real-time
object tracking system. The CCD tracker is presented and developed in
this thesis, and simulation with experimental results have been presented, through a
stand-alone implementation running both on
standard hardware and PR2.

We demonstrate the experimental results and compare it with other
segmentation or tracking algorithm regarding some aspects such as
robustness, performance and accuracy. Also qualitative
results are presented in this thesis.

The experimental analysis shows that the CCD approach is capable of
achieving high sub-pixel accuracy and robustness even in the presence
of heavy texture, clutter, partial occlusion and severe changes of
illumination. It could be used in  visual-guided robot
manipulation, robot navigation (self-localization) and target
tracking.

\section{Feature work}
\label{sec:feature}

Though the CCD algorithm is a powerful local curve-fitting algorithm,
like other computer vision algorithms, it is not a general solution to
the image segmentation and curve fitting problem. There are following
limitations or shortcomings:
\begin{itemize}
\item The current implementation strongly depends on values of pixels in
  images. It is also an algorithm relying on RGB statistics. Therefore, it
  does not work well for gray-scale images.
\item In many cases, if the automated global initialization of contours does
  not work, then manual intervention has to be executed at the beginning or
  in case of tracking loss.
\item For an object like e.g. a ring, there are two extremely similar
  contours which are very close. The separating criteria, i.e. local
  statistics, is sometimes not capable of distinguishing two contours.
\item In the process of optimization, the algorithm might converge to
  local minimum which could lead to a failure. For instance, the
  contour of the shadow is sometimes similar to the object being
  studied. That is why the CCD algorithm could not obtain perfect results in this case.
\end{itemize}

We could improve the CCD algorithm in order to make it
more stable for the application of personal robotics.
\begin{itemize}
\item Besides the RGB statistics, statistics based on the feature
  descriptors and color values in other color space could be used as
  local feature. This will make the algorithm adaptable.
\item The current 6-DOF and 8-DOF shape-space could not cope with all
  cases. When perspective effects are strong, approximation error
may be appreciated. Therefore, some more complex model representations
should be designed.
\item In order to avoid the local minima, some advanced optimization
  methods should be investigated and implemented.
\item In the CCD algorithm, we do not take into account
  the coupling of different pixels groups on perpendiculars. We also
  do not consider the statistical dependencies between successive
  images in the CCD tracker. Despite that it is more
  complicated when considering these factors, a performance and
  stability enhancement could be achieved.
\end{itemize}

Moreover, Some possible applications of the CCD approach could be
investigated:
\begin{itemize}
\item The CCD algorithm could be integrated into a more complex
  tracking framework by combining it with some other efficient
  algorithms and to be implemented in complex applications.
\item Taking into account the advantages of the CCD algorithm,
  robustness, stability and accuracy, it can be used for object
  localization and recognition.
\item Now the OpenCV library has provided support on the platform of
  Android~\cite{android}. Therefore, the optimized CCD code could be ported to
  Android system and used in mobile applications.
\end{itemize}
