%%% Local Variables: 
%%% mode: latex
%%% TeX-master: "../main"
%%% End: 

\chapter{Conclusion and  Future Work}
\label{chapter:conclusion}

\section{Conclusion}
\label{sec:con}

In this work, we have investigated the Contracting Curve Density (CCD)
algorithm and its application to personal robotics. Starting from
parametric model curve, we have analyzed the core of the algorithm and
proposed some improvements and optimizations on it. Then we applied the
algorithm on some challenging robotic perception problems, such as curve
fitting, object segmentation and tracking, and 3-D pose estimation. We
carried out some experiments on the PR2 robot which are showing satisfactory results. 

The CCD algorithm is an iterative procedure to refine a priori distribution of the model parameters to a Gaussian
approximation of the posterior distribution. It aims to extract the
contour of an object based on a priori model. A MAP estimate needs to be
evaluated by applying an effective local optimization method in order to get
the mean of model parameters and covariance matrices which govern the
posterior distribution and determine the shape of the model. Likelihood is 
obtained from local statistics of the vicinity of the expected
curve. Since the local statistics provides locally adapted criteria for
adjacent image regions seperated by a contour. This criteria is stable
and effective than those relying on homogeneous image regions or
specific edge properties. The local statistics are very
important for the CCD algorithm, we design a new weight
function for separating adjacent image regions. 
 In order to improve the performance and
reduce computational cost, a blurred curve model~\cite{hanek2004contracting} is applied as an efficient means for
iterative optimization. In the iteration process, after the
algorithm starts to converge, the area of
convergence becomes smaller and smaller. This has two great
advantages, one is that the sample points used to learn the local statistic
are scaled, thus computing cost is decreased. Another one is that it
achieves a high level sub-pixel accuracy. The decreasing rate of the
convergence area is important, if too fast, the algorithm will stop
converging, on the other hand, if too slow, it is difficult to achieve
notably performance improvement. In this thesis we have investigated the relation between
the rate and the convergence speed. In total the following set of
improvements over the original implementation is brought into our system:
\begin{itemize}
\item Use of the logistic sigmoid function instead of Gaussian error
  function to convert a curve-fitting problem to a Gaussian logistic
  regression problem.
\item Modeling the parametric curve is implemented using the quadratic or
cubic B-spline curve  to avoid the Runge phenomenon without increasing
the degree.
\item Supports for both planar affine and three-dimensional affine
  shape-space. This makes the system viewpoint invariant.
\item Full automatic initialization of contour from SIFT keypoints or
  point clouds is developed to avoid
  intervention from human.
\end{itemize}

Due to the robustness, stability and high performance, one main
application of the CCD algorithm is the development of real-time
object tracking system. The CCD tracker is presented and developed in
this thesis, and simulation with experimental results have been presented, through a
stand-alone implementation running both on
standard hardware and the PR2 Robot.

% We demonstrate the experimental results and compare it with other
% segmentation or tracking algorithm regarding some aspects such as
% robustness, performance and accuracy. Also qualitative
% results are presented in this thesis.

The experimental analysis shows that the CCD approach is capable of
achieving high sub-pixel accuracy and robustness even in the presence
of heavy texture, clutter, partial occlusion and severe changes of
illumination. It could be used in  visual-guided robot
manipulation, robot navigation (self-localization) and target
tracking.

\section{Future Work}
\label{sec:feature}

Though the CCD algorithm is a powerful local curve-fitting algorithm,
it has the following limitations or shortcomings:
\begin{itemize}
\item The current implementation strongly depends on RGB values of pixels in
  images, which is error-prone to lighting changes.% . It is also an algorithm relying on RGB statistics. Therefore, it
  % does not work well for gray-scale images.
\item In many cases, if the automated global initialization of contours does
  not work, the manual intervention has to be executed at the beginning or
  in case of tracking loss.
\item B-spline can not precisely represent many useful simple curves
  such as circles and ellipses, thus, Non-uniform Rational B-spline
  (NURBS) is required for the CCD algorithm.
\item For an object like e.g. a ring, there are two extremely similar
  contours which are very close. The separating criteria, i.e. local
  statistics, is thus not capable of distinguishing two contours.
\item In the process of optimization, the algorithm might converge to
  local minimum which could lead to a failure. For instance, the
  contour of the shadow is sometimes similar to the object being
  studied. That is why the CCD algorithm could not obtain perfect results in this case.
\end{itemize}

We propose to improve the CCD algorithm in order to make it
more stable for the application of personal robotics.
\begin{itemize}
\item Besides the RGB statistics, statistics based on the feature
  descriptors and color values in other color space could be used as
  local feature. This will make the algorithm adaptable.
\item The current 6-DOF and 8-DOF shape-space can not cope with all
  cases.%  When perspective effects are strong, approximation error
% may be appreciated. 
Therefore, some more complex model representations should be designed.
% \item In order to avoid the local minima, some advanced optimization
%   methods should be investigated and implemented.
\item In the CCD algorithm, we do not take into account
  the coupling of different pixels groups on normals. We also
  do not consider the statistical dependencies between successive
  images in the CCD tracker. Considering these factors, the performance and
  the stability enhancement could be achieved.
\end{itemize}

Lastly, the following applications of the CCD approach could be investigated:
\begin{itemize}
\item Integration of the CCD algorithm into a more complex
  tracking framework by combining it with some other efficient
  algorithms (e.g. the Lucas-Kanade method (\acronym{LKM}{Lucas-Kanade method}), the extended Kalman
  filter (\acronym{EKF}{extended Kalman filter})).
\item Now the OpenCV library has provided support on the platform of
  Android~\cite{android}. Therefore, the optimized CCD code could be ported to
  Android system and used in mobile applications.
\end{itemize}
