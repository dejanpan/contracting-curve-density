%%% Local Variables: 
%%% mode: latex
%%% TeX-master: "../main"
%%% End: 

\chapter{Conclusion and  Future Work}
\label{chapter:conclusion}

\section{Conclusion}
\label{sec:con}

In this work, we have investigated the Contracting Curve Density (CCD)
algorithm and its application on personal robotics. Starting from
parametric model curve, we have analyzed the core of the algorithm and
proposed some improvement and optimization on it. Then we apply the
algorithm on some challenging computer vision problems, such as curve
fitting, object segmentation and tracking, 3-D pose estimation.  We
delivered some experiments on PR2 and show satisfactory results. The
experimental analysis shows that the CCD approach is capable of
achieving high sub-pixel accuracy and robustness even in the presence
of heavy texture, clutter, partial occlusion and severe changes of
illumination.

The CCD algorithm is a iterative procedure to refine an
a prior distribution of the model parameters to a Gaussian
approximation of the posterior distribution. It aims to extract the
contour of a object based a prior model. A MAP estimate need to be
evaluated by apply an effective optimization methods in order to get
the mean of model parameters and covariance matrices which govern the
posterior distribution and determine the shape of model. Likelihood is 

The separation criteria are iteratively
obtained from local statistics of the curve's vicinity. Pixels, which
are intersected by the curve are modeled by a mixture of two local
distributions corresponding to the two sides of the curve. This
locally adapted statistical modeling allows for separating the two
sides of the curve with high sub-pixel accuracy even in the presence
of severe texture, shading, clutter, partial occlusion, and strong
changes of illumination.


The CCD algorithm employs a blurred curve model as a means for
iteratively optimizing the fit. The algorithm optimizes the MAP
criteria not only for a single vector of model parameters but for a
distribution of model parameter vectors defining a blurred
curve. During the iterative process not only the model parameters are
refined but also the associated covariance matrix. From the covariance
matrix the local uncertainties of the curve are obtained, which
provide the basis for the automatic and local scale selection. Our
experiments show that the CCD algorithm achieves high sub-pixel
accuracy and a large area of convergence even for challenging
scenes. Hanek (2003) proposed an extension of the CCD algorithm called
CCD tracker. The CCD tracker fits a curve to a sequence of images. The
conducted experiments show that the CCD tracker outperforms other
state-of-the-art trackers in terms of accuracy, robustness, and
runtime (Hanek, 2003).

\section{Feature work}
\label{sec:feature}

