% Created 2010-11-12 Fri 21:29
\documentclass[11pt]{article}
\usepackage[utf8]{inputenc}
\usepackage[T1]{fontenc}
\usepackage{ezfontcfg}
\usepackage[version=3]{mhchem}
\usepackage[colorlinks=true,bookmarksnumbered=true,linkcolor=blue,pdfstartview=FitH]{hyperref}

\usepackage{fixltx2e}
\usepackage{graphicx}
\usepackage{longtable}
\usepackage{float}
\usepackage{wrapfig}
\usepackage{soul}
\usepackage{textcomp}
\usepackage{marvosym}
\usepackage{wasysym}
\usepackage{latexsym}
\usepackage{amssymb}
\usepackage{amsmath}
\usepackage{amsfonts}
\usepackage{ifthen}
\usepackage{hyperref}
\usepackage{mypgf}
\providecommand{\alert}[1]{\textbf{#1}}

\title{Energy Function}
\author{Shulei ZHU}
\date{12 November 2010}

\begin{document}

\maketitle

\setcounter{tocdepth}{2}
\tableofcontents
\vspace*{1cm}

\section{Energy function}
\label{sec-1}

\begin{displaymath}
\mathcal{X}^2(\phi) = \mathrm{argmin}_{\phi} \mathcal{X}^2(\phi)
\end{displaymath}
where 
\begin{align*}
\mathcal{X}^2(\phi) = & -2\ln \left\{ {\frac{1}{{(2\pi)}^{1/2}}}
\frac{1}{|\Sigma_{\phi}^{*}|} \mathrm{exp}\{-\frac{1}{2}{(\phi-m_{\phi}^{*})^T{\Sigma_{\phi}^{*}}^{-1}(\phi-m_{\phi}^{*})}\}
\right\}  \\ & - 2 \ln \left\{ \prod_{v \in \mathcal{V}}^{N} {\frac{1}{(2\pi)^{1/2}}
\frac{1}{|\Sigma_v|} \mathrm{exp}\{-\frac{1}{2}
{\left[I_{v}-m_v(a_{v,1})\right]^T\{\Sigma_{v}(a_{v,1})\}^{-1}\left[I_{v}-m_v(a_{v,1})\right]}
}\} \right\}
\end{align*}

Where we assume $m_{\phi}^{*}$ as $\left[0 \quad 0 \quad 0 \quad 0 \quad 0 \quad 0 \right]$ and $m_{\phi}^{*}$ as a $6 \times 6$ identity
matrix, $m_v$ and $\Sigma_{v}$ represent the mean value and covariance
of a given point on the curve.

\begin{displaymath}
m_v = a_{v,1} m_{v,1} + (1- a_{v,1}) m_{v,2}
\end{displaymath}
\begin{displaymath}
\Sigma_{v} = a_{v,1} \Sigma_{v,1} + (1- a_{v,1}) \Sigma_{v,2}
\end{displaymath}

$m_{v,1}(m_{v,2})$, $\Sigma_{v,1}(\Sigma_{v,2})$ are the mean
values and covariance metrics in the positive(negative) normal
direction near the vicinity of a given point, but the formula to
calculate these four variables is very complicated, so I will not show
it. $a_{v,1}$ ,$a_{v,2}$ are the fuzzy weights of a
given point locating in side 1 and 2, N means the number of points on
the curve. $I_v$ denotes the pixel value of given point. $a_{v,1}$ and
$a_{v,2}$ are given by

\begin{displaymath}
a_{v,1} = \Sigma_{i}^{m} \left\{ \frac{1}{2}\cdot
erf(\frac{d_v(x)}{\sqrt{2}\cdot\sigma_v}) + \frac{1}{2}\right\}
\end{displaymath}

\begin{displaymath}
a_{v,2} = 1- a_{v,1}
\end{displaymath}


here $m$ means the number of points in the positive normal ($+\vec{n}$)
direction in the vicinity of a given point, there are also m points in
the negative normal direction ($-\vec{n}$), $\sigma_{v}$ is given by
\begin{displaymath}
\sigma_v^2 = \mathbf{n}_v^T\cdot
\mathbf{J}_v \cdot \mathbf{\Sigma}_{\phi}\cdot \mathbf{J}_v^T\cdot \mathbf{n}_v
\end{displaymath}
where $\mathbf{n}$ is the normal vector, $\mathbf{J}_v$ denots the Jacobian of curve, i.e. the partial derivatives
of c with respect to model parameters $\phi$ in the given point,e.g we
are given a point $(p_x,p_{y})$, and model parameters are given by
$\left[ x_0, x_1, x_2, x_3, x_4, x_5 \right]$, we can write $\mathbf{J}_v$ as

\[
\mathbf{J}_v =
\left[ {\begin{array}{cccccc}
\frac{\partial p_x}{\partial x_0}& \frac{\partial p_x}{\partial x_1}& \frac{\partial p_x}{\partial x_2}& \frac{\partial p_x}{\partial x_3}&\frac{\partial p_x}{\partial x_4} &\frac{\partial p_x}{\partial x_5}  \\
\frac{\partial p_y}{\partial x_0}& \frac{\partial p_y}{\partial x_1}& \frac{\partial p_y}{\partial x_2}& \frac{\partial p_y}{\partial x_3}&\frac{\partial p_y}{\partial x_4} &\frac{\partial p_y}{\partial x_5}  \\
 \end{array} } \right]
\]

Furthermore, The energy function can be simplified as 
\begin{align*}
\mathcal{X}^2(\phi) = & \ln{(2\pi)} + 2\ln{|\Sigma_{\phi}^{*}|} + {\phi}^T{\Sigma_{\phi}^{*}}^{-1}\phi
 \\ & + N\ln{2\pi} + 2N\Sigma_{v \in \mathcal{V}}^{N}{\ln{|\Sigma_v|}} + \Sigma_{v \in \mathcal{V}}^{N} 
\left\{{\left[I_{v}-m_v(a_{v,1})\right]^T\{\Sigma_{v}(a_{v,1})\}^{-1}\left[I_{v}-m_v(a_{v,1})\right]}\right\}
\end{align*}
\section{Compute the first order derivative of energy function}
\label{sec-2}

\begin{align*}
\nabla_{\phi}\{{\mathcal{X}^2(\phi)}\} = & 0 + 0
+ \{{\Sigma_{\phi}^{*}}^{-1}\}^{T}{\phi} + {\Sigma_{\phi}^{*}}^{-1}\phi
 \\ & + 0 + 0 + \Sigma_{v \in \mathcal{V}}^{N} \nabla_{\phi}\left\{
{\left[I_{v}-m_v(a_{v,1})\right]^T\{\Sigma_{v}(a_{v,1})\}^{-1}\left[I_{v}-m_v(a_{v,1})\right]}\right\}
\end{align*}

\textbf{The question is how to calculate the last term in the above formula}.

PS. You can find the expression of $a_{v,1}$ in the first part.

\end{document}
