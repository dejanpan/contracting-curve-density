% Created 2010-09-03 Fri 12:39
\documentclass[11pt]{article}
\usepackage[utf8]{inputenc}
\usepackage[T1]{fontenc}
\usepackage{ezfontcfg}
\usepackage[version=3]{mhchem}
\usepackage[colorlinks=true,bookmarksnumbered=true,linkcolor=blue,pdfstartview=FitH]{hyperref}

\usepackage{fixltx2e}
\usepackage{graphicx}
\usepackage{longtable}
\usepackage{float}
\usepackage{wrapfig}
\usepackage{soul}
\usepackage{textcomp}
\usepackage{marvosym}
\usepackage{wasysym}
\usepackage{latexsym}
\usepackage{amssymb}
\usepackage{amsmath}
\usepackage{amsfonts}
\usepackage{ifthen}
\usepackage{hyperref}
\usepackage{mypgf}
\providecommand{\alert}[1]{\textbf{#1}}

\title{CCD algorithms}
\author{Shulei ZHU}
\date{03 September 2010}

\begin{document}

\maketitle


 Steps(for segmentation)

\section{Initialization}
\label{sec-1}

   use b-spline to make the contour $\mathbb{C}$ of one object
\section{Compute local, mixed color statistics:}
\label{sec-2}


Collect all color pixels along the contour $\mathbb{C}$, calculate mean value $M\_in$, $M\_out$ and  covariance $Covar\_in$,
$Covar\_out$ (subscript \textbf{in} means inside the contour, \textbf{out} means
outside the contour)
\section{Calculate the cost function (likelihood function of local statistics)}
\label{sec-3}

\begin{itemize}
\item Compute the color gradient
\item Compute the color Jacobian matrix
\end{itemize}
\section{Apply the Levenberg–Marquardt algorithm to the cost function}
\label{sec-4}

   get $\delta p$, update the pose of contour $\mathbb{C} \rightarrow
   p+\delta p$ 
\section{if $\delta p$ is smaller than a given threshold, then stop otherwise, go to step 2.}
\label{sec-5}

\end{document}
