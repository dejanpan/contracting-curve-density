% Abstract for the TUM report document
% Included by MAIN.TEX


\clearemptydoublepage
\phantomsection
\addcontentsline{toc}{chapter}{Abstract}        





\vspace*{2cm}
\begin{center}
{\Large \bf Abstract}
\end{center}
\vspace{1cm}

This master's thesis investigates an extended and optimized
implementation of a state-of-the-art local curve fitting algorithm,
the the Contracting Curve Density (CCD).
In particular, we investigate
the application of the CCD algorithm and its variant the CCD tracker
on personal robotics (e.g. PR2).

We developed an application program based on the OpenCV library, which
is a cross-platform computer vision library. Based on the
design conception of the CCD algortihm, some
additional features are integrated into the developed program in order to
achieve stability and robustness. First, the quadratic (degree-2) or
cubic (degree-3) B-spline curve is used to model the parametric curve
to avoid the Runge phenomenon without increasing the degree of the
B-spline. secondly, the system supports both planar affine (6-DOF) and
non-planar affine (8-DOF) shape-space. The extended affine space can avoid
mismatch of curve cauased by minor viewpoint changes. Moreover, in
order to avoid manual intervention from the user, the developed system
also supports robust global initialization modules based on both keypoint
feature matching and 3-Dimensional point cloud. At last, the CCD
algorithm solves the curve fitting problem by converting it to a
problem in pattern recognition, and maximum a posteriori probability
(MAP) estimate is an important but time-cost step. We proposed some
changes and optimization based on it.

The developed system mainly consists of two functional parts, the CCD
algorthm to fit model curve in still images and the CCD tracker to
track model in sequential-frame video. We apply the CCD algorithm to
serval image segmentation and object tracking problems. Some experiments with RGB
images and videos are delivered on Personal Robot 2 (PR2), and the
performance is examined on two tasks listed above.  It is shown that
the CCD algorithm achieves robustness and subpixel accuracy even in the
presence of clutter, partial occlusion, and changes of
illumination. We present results for some curve-ftting problems, such
as image segmentation, 3-D pose estimation, object tracking based on
both manual intervention and automatical initialization.
















