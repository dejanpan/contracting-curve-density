% Abstract for the TUM report document
% Included by MAIN.TEX


\clearemptydoublepage
\phantomsection
\addcontentsline{toc}{chapter}{Abstract}        





\vspace*{2cm}
\begin{center}
{\Large \bf Abstract}
\end{center}
\vspace{1cm}

% This master's thesis investigates an extended and optimized
% implementation of a state-of-the-art local curve fitting algorithm,
% the Contracting Curve Density (CCD).
% In particular, we investigate
% the application of the CCD algorithm and its variant the CCD tracker
% on personal robotics (e.g. PR2).

% We developed an application based on the OpenCV library, which
% is a cross-platform computer vision library. Based on the
% design conception of the CCD algortihm, some
% additional features are integrated into the developed program in order to
% achieve stability and robustness. Firstly, we use the logistic sigmoid
% function instead of Gaussian error function, thus completely convert a
% curve-fitting problem to a Gaussian logistic regression problem in the
% field of pattern recognition. Secondly, the quadratic (degree-2) or
% cubic (degree-3) B-spline curve is used to model the parametric curve
% to avoid the Runge phenomenon without increasing the degree of the
% B-spline. Thirdly, the system supports both planar affine (6-DOF) and
% non-planar affine (8-DOF) shape-space. The extended affine space can avoid
% curve mismatching caused by minor viewpoint changes. Moreover, in
% order to avoid manual intervention from the user, the developed system
% also supports robust global initialization modules based on both keypoint
% feature matching and 3-Dimensional point cloud. At last, the CCD
% algorithm solves the curve fitting problem by converting it to a
% pattern recognition problem, and because maximum a posteriori probability
% (MAP) estimate is an important but time-cost step. We proposed some
% changes and optimization on it.

% The developed system mainly consists of two functional parts, the CCD
% algorithm to fit model curve in still images and the CCD tracker to
% track model in sequential-frame video. We apply the CCD algorithm to
% serval image segmentation and object tracking problems. Some experiments with RGB
% images and videos are delivered on Personal Robot 2 (PR2), and the
% performance is examined on two tasks listed above.  It is shown that
% the CCD algorithm achieves robustness and subpixel accuracy even in the
% presence of clutter, partial occlusion, and changes of
% illumination. We present results for some curve-ftting problems, such
% as image segmentation, 3-D pose estimation, object tracking based on
% both manual intervention and automatical initialization.

This thesis investigates an extended and optimized
implementation of the state-of-the-art local curve fitting algorithm
named Contracting Curve Density (CCD) algorithm, originally developed 
by Hanek et al. In particular, we investigate its application  
in the field of personal robotics for the tasks such as the segmentation
of objects in clutter and the tracking of objects. 

The CCD algorithm can be best described as follows. Given one or multiple images as input
data and a parametric curve model with a priori distribution of model
parameters, through curve-fitting process, we estimate the model
parameters which determine the approximation of the posterior
distribution in order to make the curve models best matching the image data.
In order to improve the stability, accuracy and robustness over the original
implementation we introduce the following improvements. Firstly, we use the logistic sigmoid
function instead of a Gaussian error function which renders a
curve-fitting problem as a Gaussian logistic regression problem known in the
field of pattern recognition. Secondly, a quadratic or
a cubic B-spline curve is used to model the parametric curve
to avoid the Runge phenomenon without increasing the degree of the
B-spline. Thirdly, the system supports both planar affine (6-DOF) and
three-dimensional affine (8-DOF) shape-space. The latter affine space can avoid
curve mismatching caused by major viewpoint changes. Lastly, in
order to avoid manual intervention by the user, the developed system
also supports robust global initial curve initialization modules based on both keypoint
feature matching and back-projections from the 3D point clouds. 

The developed system mainly consists of two functional parts, the CCD
algorithm to fit the model curve in still images and the CCD tracker to
track the model in the videos. We demonstrate algorithm's working 
in various scenes using handheld camera and the cameras from the 
PR2 robot. Achieved results show that the CCD algorithm achieves 
robustness and sub-pixel accuracy even in the presence of clutter, 
partial occlusion, and changes of illumination. 












